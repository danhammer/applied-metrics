% Created 2012-10-14 Sun 12:34
\documentclass{article}
\usepackage[utf8]{inputenc}
\usepackage[T1]{fontenc}
\usepackage{fixltx2e}
\usepackage{graphicx}
\usepackage{longtable}
\usepackage{float}
\usepackage{wrapfig}
\usepackage{soul}
\usepackage{textcomp}
\usepackage{marvosym}
\usepackage{wasysym}
\usepackage{latexsym}
\usepackage{amssymb}
\usepackage{hyperref}
\tolerance=1000
\usepackage{mathrsfs}
\usepackage{graphicx}
\usepackage{subfigure}
\usepackage[margin=1in]{geometry}
\RequirePackage{fancyvrb}
\DefineVerbatimEnvironment{verbatim}{Verbatim}{fontsize=\small,formatcom = {\color[rgb]{0.1,0.2,0.9}}}
\providecommand{\alert}[1]{\textbf{#1}}

\title{}
\author{}
\date{\today}
\hypersetup{
  pdfkeywords={},
  pdfsubject={},
  pdfcreator={Emacs Org-mode version 7.8.06}}

\begin{document}



\setlength{\parindent}{0in}

\textbf{Interpretation of Coefficients and R Review} \hfill
\textbf{ARE213}: Section 01 \\ \\

The purpose of this section is twofold: (1) Basic review of the
interpretation of coefficients on common, linear models that we will
encounter in class. (2) Introduction and quick review of useful \texttt{R}
code to help with problem sets.  The section notes are an open-source,
Github project, which can be found here.  The text is an \texttt{org-mode}
document, which can be compiled as HTML or \LaTeX.

\section*{Interpretation of Coefficients}
\label{sec-1}

\end{document}
